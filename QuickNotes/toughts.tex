\documentclass[a3paper]{article}

\usepackage{amsmath, amssymb, braket}
\usepackage{graphicx}
\usepackage[ruled, lined, boxed]{algorithm2e}
\usepackage[colorlinks]{hyperref}
\usepackage{parcolumns}
\usepackage{mathpazo}

\newcommand{\LTL}{{\sf LTL}}
\newcommand{\CTL}{{\sf CTL}}
\newcommand{\CTLs}{{\sf CTL$^\star$}}

% "intuitive" meaning

\newcommand{\Intuit}[1]{``\emph{#1}''}

% LTL/CTL Operators

\newcommand{\Op}[1]{\mathcal{\sf \bf #1}}

\newcommand{\X}[1]{\Op{X}#1}
\newcommand{\U}[2]{({#1}\Op{U}{#2})}
\newcommand{\G}[1]{\Op{G}#1}
\newcommand{\F}[1]{\Op{F}#1}

\newcommand{\E}[1]{\mathcal{\Op{E}}#1}
\newcommand{\A}[1]{\mathcal{\Op{A}}#1}

\newcommand{\EX}[1]{\E{\X{#1}}}
\newcommand{\EU}[2]{\E({\U{#1}{#2}})}
\newcommand{\EG}[1]{\E{\G{#1}}}
\newcommand{\EF}[1]{\E{\F{#1}}}

\newcommand{\AX}[1]{\A{\X{#1}}}
\newcommand{\AU}[2]{\A{(\U{#1}{#2}})}
\newcommand{\AG}[1]{\A{\G{#1}}}
\newcommand{\AF}[1]{\A{\F{#1}}}

% Denotation, Symbolic & stuff

\newcommand{\Denot}[1]{\left[{#1}\right]}
\newcommand{\Symb}[1]{\xi\left({#1}\right)}
\newcommand{\PreImage}[1]{\mbox{\emph{PreImage}}\left({#1}\right)}
\newcommand{\Image}[1]{\mbox{\emph{Image}}\left({#1}\right)}

% Fixed points

\newcommand{\MinFP}[2]{\mu{#1}\:.\:{#2}}
\newcommand{\MaxFP}[2]{\nu{#1}\:.\:{#2}}

% General logic

\newcommand{\imply}{\rightarrow}
\newcommand{\Imply}{\Rightarrow}
\newcommand{\IterAnd}[3]{\bigwedge_{#1}^{#2}\left({#3}\right)}
\newcommand{\IterOr}[3]{\bigvee_{#1}^{#2}\left({#3}\right)}
\newcommand{\lOr}{\vee}
\newcommand{\lAnd}{\wedge}
\newcommand{\Entails}{\vDash}
\newcommand{\Tuple}[1]{\left<{#1}\right>}

% Constants

\newcommand{\Good}{\mathcal{{\tt good}}}
\newcommand{\Bad}{\mathcal{{\tt bad}}}



\title{Toughts}
\author{Giovanni Simoni}

\begin{document}

    \maketitle

    \section{About expressiveness of \CTL\ and \LTL}

        $\LTL$ is about paths, while $\CTL$ is about single steps in which
        I can chose, at any time, the direction I'm gonna take on the next
        step. In the $\LTL$ approach, if I've chosen a path I cannot
        change my mind.

        \subsection{Express \CTL\ with \LTL}

            \begin{itemize}

            \item   An expression using existential quantifiers in
                    \CTL\ requires, in order to be satisfied, to be true
                    for a subset of the available paths.

            \item   All \LTL\ formulae implicitly require an universal
                    quantification among all possible paths, thus \LTL\
                    formula, in order to be true, must be satisfied by all
                    possible paths.

            \end{itemize}

        \subsubsection{Express \LTL\ with \CTL}

            \begin{itemize}

            \item   A typical check we want to achieve is about fairness
                    properties:
                    \[
                    \left(
                        \IterAnd{f \in \Set{ f_1 \cdots f_n }}{}{\G\F f}
                    \right)
                    \imply \psi
                    \]

            \item   $\G{\F{\phi}}$ in \CTL\ becomes $\AG{\AF{\phi}}$;

            \item   Since $a \imply b$ is defined as $\neg a \lOr b$ we
                    have a negation of $\G{\F{\phi}}$ terms;

            \item   $\neg \G{\F{\phi}} = \F{\G{\neg \phi}}$, namely
                    \Intuit{in each path, after a certain point, $\phi$
                    will never hold again};

            \item   $\neg \AG{\AF{\phi}} = \EF{\EG{\neg \phi}}$, namely
                    \Intuit{there's a path among many in which $\phi$ is
                    not holding} (much less restrictive than $\F{\G{\neg
                    \phi}}$, which requires $\neg \phi$ to hold in all
                    paths).

            \end{itemize}

    \section{About SCC-based algorithm for $\Ef{\G{\varphi}}{f}$}

        The Fair Strongly Connected Components are found in a Kripke Model
        which is limited to the states in which $\varphi$ holds.

        The fairness comes directly from the fact we used a fair Kripke
        Model, while the SCC gives guarantees about existence of a path
        for which $\varphi$ always holds. At this point we do the union
        $C$ of all SCC, and we compute $\E{\U{\Denot{\varphi}}{C}}$.

        The point is that $C$ ensures both membership in $\Denot{\varphi}$
        and fairness, while $\Denot{\varphi}$ doesn't give fairness. Note
        that we require fair states to be touched infinitely often, but
        not to be globally into fair states.

\end{document}

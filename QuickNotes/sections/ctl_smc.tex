Basic idea: use formulae to represent set of states.
\begin{itemize}

    \item   Each state of the Kripke Model is mapped to a truth assignment
            over $n$ variables;

    \item   Each formula $p$ encodes the set of states for which the
            formula holds.

\end{itemize}

\subsection{OBDDs}

    \begin{itemize}

    \item   Fixed order for variables, imposed a priori:
        \begin{itemize}

        \item   Such ordering is crucial w.r.t performances;

        \item   Structure follows the principle of the DPLL for SAT
                following the ordering:

            \begin{itemize}
            \item   Starting from the first variable $A_1$ we build two
                    formulae equal to the original one, the first with all
                    occurrences of $A_1$ replaced by \emph{True}, the
                    second with all occurrences of $A_1$ replaced by
                    \emph{False};
            \item   The structure handles efficiently Shannon's expansion.
            \end{itemize}

        \item   The graph {\bf construction} follows the incremental
                building algorithm (using the \emph{ite}
                (\Intuit{If-Then-Else}) as basic item and following
                the variables ordering).

        \end{itemize}

    \item   Reduction technique:
        \begin{itemize}
        \item   The structure in principle is tree-shaped, with an
                exponential number of nodes (a level for each variable,
                $2^n$ nodes where $n$ is the number of variables.
        \item   Sharing sub-nodes;
        \item   Removing redundancies (nodes with the same left and right
                children collapse into one).
        \end{itemize}

    \item   Semantics: one-to-one mapping with logic formulae, constant
            time comparison, validity and unsatisfiability (the complexity
            is moved into the construction phase).

    \end{itemize}

\subsection{Representation of Kripke Model}

    \begin{itemize}

    \item   A set of states $Q \subseteq S$ corresponds to a formula
            $\Symb{Q}$ such that
            \[ \forall s \in Q \:.\: s \Entails \Symb{Q} \]

        \begin{itemize}

        \item   $\Symb{}$ maps sets (typically the output of the
                \emph{denotation} function $\Denot{-}$) into symbolic
                formulas;

        \item   Naive construction (exponential size):
                $\Symb{Q} = \IterOr{s \in Q}{}{\Symb{s}}$;

        \item   Structurally is mapped on set operators (although the set
                of denotations are never explicitly constructed).

        \end{itemize}

    \item   Transition relation is a single formula with
            twice the number of variables:
        \[
        \Symb{R} = \IterOr{\Tuple{s, s^\prime} \in R}{}
                                   {\Symb{s} \lAnd \Symb{s^\prime}}
        \]

    \item   Symbolic Pre-image operation of a set $P$:
        \[
        \PreImage{P, R}[V] = \exists V^\prime \:.\: \Symb{P}[V^\prime]
                             \lAnd \Symb{R}[V, V^\prime]
        \]
        \begin{itemize}

        \item   $\mu \cup \mu^\prime \Entails \Symb{P}[V^\prime]
                            \lAnd \Symb{R}[V, V^\prime]$
                Where $\mu^\prime$ is the model for the next state, giving
                $V^\prime$ as assignment of variables.

        \item   Operatively it corresponds to an \emph{or} of all possible
                assignments $V^\prime$ for the next-state variables (we
                search for current states mapped into $P$ as next states).

        \end{itemize}

    \item   Symbolic Image operation of a set $P$:
        \[
        \Image{P,R}[V'] = \exists V \:.\: \Symb{P}[V] \lAnd
                          \Symb{R}[V, V^\prime]
        \]
        \begin{itemize}

        \item   Operatively it corresponds to an \emph{or} of all possible
                assignments $V$ for the current-state variables (we search
                for next states mapped into $P$ as current states).

        \end{itemize}

    \item   Fixed-point computations can be built in the same way you do
            for non-symbolic \CTL;

    \item   OBDDs are used for logic operations.

    \end{itemize}

\subsection{Checking $M \Entails \psi$}

    \begin{itemize}

    \item   In non-symbolic model checking: $\Denot{I} \subseteq
            \Denot{\psi}$;

    \item   In symbolic model checking: $\Symb{I} \imply
            \Symb{\Denot{\psi}}$.
        \begin{itemize}
        \item   Note that computation of $\Denot{\psi}$ requires fixed
                point computation;
        \item   However we don't compute the denotation: we work directly
                with symbolic formulas, thus the fixed point is reached
                through symbolic preimages.
        \end{itemize}

    \end{itemize}

    \Note{Formulas are assumed to be written in terms of $\neg a$,
          $a \lAnd b$, $\E{\X{a}}$, $\E{\G{a}}$, $\E{\U{a}{b}}$}

    \subsubsection{Checking invariants $\AG{\neg\Bad}$}
        \begin{itemize}
        \item   Backward strategy: apply Pre-images to $\Bad$ until the
                set $I$ is intersected or a fixed point is reached;
        \item   Forward strategy: apply Images to the set $I$ until a
                $\Bad$ or a fixed point is reached:
            \begin{itemize}
            \item   Optimization: keep incremental set of images but apply
                    Image function only to the frontier (namely the
                    intersection between the previous image and the
                    negation of the incremental set itself);
            \item   Counter-example generation:
                \begin{itemize}
                \item   Keep track of fronteers: $F_1 \cdots F_n$;
                \item   We reached bad state if $F_n \cap \Bad
                        \neq\emptyset$;
                \item   Find $t_{n-1} = F_{n-1} \cap \PreImage{F_n \cap
                        \Bad}$.
                \item   Continue to $t_0$ and get Counter-example;
                \item   Use magnets to make it simpler.
                \end{itemize}
            \end{itemize}
        \end{itemize}

